\documentclass[aspectratio=169,xcolor=dvipsnames, t]{beamer}
\usetheme{roubaix}

\addbibresource{main.bib}

%----------------------------------------------------------------------------------------
%	TITLE PAGE CONFIGURATION
%----------------------------------------------------------------------------------------
\title[short title]{Long Title of the Presentation} 
% The short title appears at the bottom of every slide, the full title is only on the title page
% Subtitle is used as the footline text
\subtitle{author et al., [conf name+year]}
\author[Surname]{Name Surname}
\institute[lab]{Name of the lab and/or institution}
\date{\today} % Date, can be changed to a custom date


\begin{document}
%----------------------------------------------------------------------------------------
%	PRESENTATION SLIDES
%----------------------------------------------------------------------------------------
\begin{frame}[plain]
    \maketitle
\end{frame}
\begin{frame}[plain]{Overview}
    \tableofcontents
\end{frame}

\section{section 1}
\begin{frame}[t]{This is a long slide title with details}
    \framesubtitle{with a subtitle}
    \begin{itemize}
        \item Lorem ipsum dolor sit amet, consectetur adipiscing elit\footcite{adamou2004}
        \item Aliquam blandit faucibus nisi, sit amet dapibus enim tempus eu
        \item Nulla commodo, erat quis gravida posuere, elit lacus lobortis est, quis porttitor odio mauris at libero
        \item Nam cursus est eget velit posuere pellentesque\footcite{kiefer2019}
        \item Vestibulum faucibus velit a augue condimentum quis convallis nulla gravida
    \end{itemize}
\end{frame}
\section{Long section two title}
\begin{frame}[t]{Another long slide with details}
    Test of the template I'm writing. In this slide, the $(u_s, p)$ formulation for the Biot model is given as 
    \begin{equation}
        \begin{aligned}
            \nabla \cdot \boldsymbol{\hat{\sigma}}^s + \tilde \rho \omega^2 \boldsymbol u^s + \tilde \gamma \boldsymbol \nabla p &= 0, \\
            \frac{\nabla^2 p}{\tilde \rho_{22} \omega^2} - \frac{\tilde \gamma}{\phi^2} \nabla \cdot \boldsymbol u^s + \frac{1}{\tilde R}p &= 0,
        \end{aligned}
        \label{eq:up_motion_equations}
    \end{equation}    
\end{frame}
\section{un titre de section}
\begin{frame}{Blocks of Highlighted Text}
    In this slide, some important text will be \alert{highlighted} because it's important.
    \begin{block}{Block}
        Sample text
    \end{block}

    \begin{alertblock}{Alertblock}
        Sample text in red box
    \end{alertblock}

    \begin{examples}
        Sample text in green box. The title of the block is ``Examples".
    \end{examples}
\end{frame}

\begin{frame}[t]{Test with columns }
    \begin{columns}[c]

        \column{.45\textwidth} 
        \textbf{Heading}
        \begin{enumerate}
            \item item 1 
            \item item 2
            \item and another item
        \end{enumerate}

        \column{.5\textwidth}
        Lorem ipsum dolor sit amet, consectetur adipiscing elit. Integer lectus nisl, ultricies in feugiat rutrum, porttitor sit amet augue. Aliquam ut tortor mauris. Sed volutpat ante purus, quis accumsan dolor.

    \end{columns}
\end{frame}
\end{document}
